\documentclass[journal,12pt,twocolumn]{IEEEtran}
%
\usepackage{setspace}
\usepackage{gensymb}
%\doublespacing
\singlespacing
\usepackage{float}
%\usepackage{graphicx}
%\usepackage{amssymb}
%\usepackage{relsize}
\usepackage[cmex10]{amsmath}
%\usepackage{amsthm}
%\interdisplaylinepenalty=2500
%\savesymbol{iint}
%\usepackage{txfonts}
%\restoresymbol{TXF}{iint}
%\usepackage{wasysym}
\usepackage{amsthm}
%\usepackage{iithtlc}
\usepackage{mathrsfs}
\usepackage{txfonts}
\usepackage{stfloats}
\usepackage{bm}
\usepackage{cite}
\usepackage{cases}
\usepackage{subfig}
%\usepackage{xtab}
\usepackage{longtable}
\usepackage{multirow}
%\usepackage{algorithm}
%\usepackage{algpseudocode}
\usepackage{enumitem}
\usepackage{mathtools}
\usepackage{tikz}
\usepackage{circuitikz}
\usepackage{verbatim}
%\usepackage{tfrupee}
\usepackage[breaklinks=true]{hyperref}
%\usepackage{stmaryrd}
\usepackage{tkz-euclide} % loads  TikZ and tkz-base
%\usetkzobj{all}
\usepackage{listings}
    \usepackage{color}                                            %%
    \usepackage{array}                                            %%
    \usepackage{longtable}                                        %%
    \usepackage{calc}                                             %%
    \usepackage{multirow}                                         %%
    \usepackage{hhline}                                           %%
    \usepackage{ifthen}                                           %%
  %optionally (for landscape tables embedded in another document): %%
    \usepackage{lscape}     
\usepackage{multicol}
\usepackage{chngcntr}
%\usepackage{enumerate}

%\usepackage{wasysym}
%\newcounter{MYtempeqncnt}
\DeclareMathOperator*{\Res}{Res}
%\renewcommand{\baselinestretch}{2}
\renewcommand\thesection{\arabic{section}}
\renewcommand\thesubsection{\thesection.\arabic{subsection}}
\renewcommand\thesubsubsection{\thesubsection.\arabic{subsubsection}}

\renewcommand\thesectiondis{\arabic{section}}
\renewcommand\thesubsectiondis{\thesectiondis.\arabic{subsection}}
\renewcommand\thesubsubsectiondis{\thesubsectiondis.\arabic{subsubsection}}

% correct bad hyphenation here
\hyphenation{op-tical net-works semi-conduc-tor}
\def\inputGnumericTable{}                                 %%

\lstset{
%language=C,
frame=single, 
breaklines=true,
columns=fullflexible
}
%\lstset{
%language=tex,
%frame=single, 
%breaklines=true
%}

\begin{document}
%


\newtheorem{theorem}{Theorem}[section]
\newtheorem{problem}{Problem}
\newtheorem{proposition}{Proposition}[section]
\newtheorem{lemma}{Lemma}[section]
\newtheorem{corollary}[theorem]{Corollary}
\newtheorem{example}{Example}[section]
\newtheorem{definition}[problem]{Definition}
%\newtheorem{thm}{Theorem}[section] 
%\newtheorem{defn}[thm]{Definition}
%\newtheorem{algorithm}{Algorithm}[section]
%\newtheorem{cor}{Corollary}
\newcommand{\BEQA}{\begin{eqnarray}}
\newcommand{\EEQA}{\end{eqnarray}}
\newcommand{\define}{\stackrel{\triangle}{=}}

\bibliographystyle{IEEEtran}
%\bibliographystyle{ieeetr}


\providecommand{\mbf}{\mathbf}
\providecommand{\pr}[1]{\ensuremath{\Pr\left(#1\right)}}
\providecommand{\qfunc}[1]{\ensuremath{Q\left(#1\right)}}
\providecommand{\sbrak}[1]{\ensuremath{{}\left[#1\right]}}
\providecommand{\lsbrak}[1]{\ensuremath{{}\left[#1\right.}}
\providecommand{\rsbrak}[1]{\ensuremath{{}\left.#1\right]}}
\providecommand{\brak}[1]{\ensuremath{\left(#1\right)}}
\providecommand{\lbrak}[1]{\ensuremath{\left(#1\right.}}
\providecommand{\rbrak}[1]{\ensuremath{\left.#1\right)}}
\providecommand{\cbrak}[1]{\ensuremath{\left\{#1\right\}}}
\providecommand{\lcbrak}[1]{\ensuremath{\left\{#1\right.}}
\providecommand{\rcbrak}[1]{\ensuremath{\left.#1\right\}}}
\theoremstyle{remark}
\newtheorem{rem}{Remark}
\newcommand{\sgn}{\mathop{\mathrm{sgn}}}
\providecommand{\abs}[1]{\left\vert#1\right\vert}
\providecommand{\res}[1]{\Res\displaylimits_{#1}} 
\providecommand{\norm}[1]{\left\lVert#1\right\rVert}
%\providecommand{\norm}[1]{\lVert#1\rVert}
\providecommand{\mtx}[1]{\mathbf{#1}}
\providecommand{\mean}[1]{E\left[ #1 \right]}
\providecommand{\fourier}{\overset{\mathcal{F}}{ \rightleftharpoons}}
%\providecommand{\hilbert}{\overset{\mathcal{H}}{ \rightleftharpoons}}
\providecommand{\system}{\overset{\mathcal{H}}{ \longleftrightarrow}}
	%\newcommand{\solution}[2]{\textbf{Solution:}{#1}}
\newcommand{\solution}{\noindent \textbf{Solution: }}
\newcommand{\cosec}{\,\text{cosec}\,}
\providecommand{\dec}[2]{\ensuremath{\overset{#1}{\underset{#2}{\gtrless}}}}
\newcommand{\myvec}[1]{\ensuremath{\begin{pmatrix}#1\end{pmatrix}}}
\newcommand{\mydet}[1]{\ensuremath{\begin{vmatrix}#1\end{vmatrix}}}
%\numberwithin{equation}{section}
%\numberwithin{equation}{subsection}
%\numberwithin{problem}{section}
%\numberwithin{definition}{section}
\makeatletter
\@addtoreset{figure}{problem}
\makeatother

\let\StandardTheFigure\thefigure
\let\vec\mathbf
%\renewcommand{\thefigure}{\theproblem.\arabic{figure}}
\renewcommand{\thefigure}{\theproblem}
%\setlist[enumerate,1]{before=\renewcommand\theequation{\theenumi.\arabic{equation}}
%\counterwithin{equation}{enumi}


%\renewcommand{\theequation}{\arabic{subsection}.\arabic{equation}}

\def\putbox#1#2#3{\makebox[0in][l]{\makebox[#1][l]{}\raisebox{\baselineskip}[0in][0in]{\raisebox{#2}[0in][0in]{#3}}}}
     \def\rightbox#1{\makebox[0in][r]{#1}}
     \def\centbox#1{\makebox[0in]{#1}}
     \def\topbox#1{\raisebox{-\baselineskip}[0in][0in]{#1}}
     \def\midbox#1{\raisebox{-0.5\baselineskip}[0in][0in]{#1}}

\title{Convex-Optimization}
\maketitle
\begin{enumerate}
\item There are three food items available, corn, milk, and bread. The table contains,
the cost per serving, the amount of Vitamin A per serving, and the number of
calories per serving for each food item. Also, there are restrictions on the total
number of calories (between 2000 and 2250) and the total amount of Vitamin A
(between 5000 and 50,000) intake in the diet.The goal of the diet problem is to
select a set of food items that will satisfy a set of daily nutritional requirement
at minimum cost. The maximum number of servings is 10 per food item. Table 1 shows all the content per serving.\\
\begin{table}[H]
 \centering
 \resizebox{\columnwidth}{!}{
 \begin{tabular}{ |c|c|c|c| } 
 \hline
 Food & Cost & Vitamin A  & calories  \\
 \hline
Corn & 0.18 USD & 107 & 72\\
 \hline
 Milk & 0.23 USD & 500 & 121\\
 \hline
 Wheat Bread & 0.05 USD &0 &65\\
 \hline
\end{tabular}}
 \caption{}
 \end{table}
\solution\\ 
Consider,\\
\begin{table}[H]
 \centering
 \resizebox{\columnwidth}{!}{
 \begin{tabular}{ |c|c|c| } 
 \hline
 Description & Parameter & Value  \\
 \hline
Cost per serving & $\vec{c}$ & $\myvec{0.18\\0.23\\0.05}$ \\
 \hline
 Vitamin A per serving &$ \vec{v} $&$ \myvec{107\\500\\0}$\\ 
 \hline
 Calories per serving & $\vec{u}$ &$\myvec{72\\121\\65}$\\ 
 \hline
 No. of servings &$\vec{x}$ &$ \myvec{x\\y\\z}$\\
 \hline
\end{tabular}}
 \caption{}
 \end{table}
Objective function 
\begin{align}
z = \min_\vec{x} \vec{c}^T\vec{x}
\end{align}
Constraints
\begin{align}
&5000 \leq 107x+500y+0z \leq 50000\\
&2000 \leq 72x+121y+65z \leq 2250\\
&0 \leq x \leq 10\\
&0 \leq y \leq 10\\
&0 \leq z \leq 10
\end{align}
Writing all the constraints in the matrix form
\begin{align}
&\vec{p}\vec{x} = \vec{q}\\
&\myvec{107&500&0\\-107&-500&0\\72&121&65\\-72&-121&-65\\1&0&0\\0&1&0\\0&0&1\\-1&0&0\\0&-1&0\\0&0&-1}\vec{x}=\myvec{50000\\ -5000\\ 2250\\ -2000\\ 10\\ 10\\ 10\\ 0\\ 0\\ 0}
\end{align}
By providing the objective function and constraints to cvxpy, we get the optimal cost (z) and optimal no.of servings $(\vec{x})$.\\
cvxpy code,
\begin{lstlisting}
https://github.com/gadepall/EE5606-optimization/codes/opt_1.py
\end{lstlisting}
From cvxpy, we get
\begin{align}
& z= 3.16  \text{USD}\\
& \vec{x} = \myvec{2\\10\\10}
\end{align}
\item Funding an expense stream. Your task is to fund an expense stream over n time periods.
We consider an expense stream $\vec{e}$ $\epsilon$ $R^n$, so that $e_t$ is our expenditure at time t.
One possibility for funding the expense stream is through our bank account. At time
period t, the account has balance $b_t$ and we withdraw an amount $w_t$. The value of our
bank account accumulates with an interest rate $\rho$ per time period, less withdrawals
\begin{align}
b_{t+1} = (1+ \rho)b_t - w_t \nonumber
\end{align}
We assume the account value must be nonnegative, so that $b_t \geq$ 0 for all t.
We can also use other investments to fund our expense stream, which we purchase at
the initial time period t = 1, and which pay out over the n time periods. The amount
each investment type pays out over the n time periods is given by the payout matrix
$\vec{P}$, defined so that $P_{tj}$ is the amount investment type j pays out at time period t per
dollar invested. There are m investment types, and we purchase $x_j \geq$ 0 dollars of
investment type j. In time period t, the total payout of all investments purchased is
therefore given by $({P x})_{t}$.
In each time period, the sum of the withdrawals and the investment payouts must
cover the expense stream, so that
\begin{align}
w_t + ({P x})_t \geq e_t \nonumber
\end{align}
for all t = 1, . . . , n.
The total amount we invest to fund the expense stream is the sum of the initial account
balance, and the sum total of the investments purchased: $b_1 + 1^T \vec{x}$.\\
Using the data in \textbf{expense stream data.$^*$,}Show that the minimum initial investment that funds the expense stream can be
found by solving a convex optimization problem.\\
\solution\\
Consider,
\begin{table}[H]
 \centering
 \resizebox{\columnwidth}{!}{
 \renewcommand{\arraystretch}{2}
 \begin{tabular}{ |c|c|c| } 
 \hline
 Description & Parameter & Value  \\
 \hline
Payout Matrix & $\vec{P}_{nxm} $ & from expense stream data.py  \\
 \hline
 expense stream & $\vec{e}_{nx1}$ &from expense stream data.py  \\ 
 \hline
 Interest rate & $\rho$ &from expense stream data.py \\ 
 \hline
 No. of investments & m &from expense stream data.py \\
 \hline
 Time period & n &from expense stream data.py \\
 \hline
  Present Bank balance & $\vec{b}_{n x 1}$ & $?$ \\
 \hline
Bank withdrawals & $\vec{w}_{n x 1}$ & $?$ \\
 \hline
Investments purchased & $\vec{x}_{m x 1}$ & $?$ \\
 \hline
Total payable for investments purchaced & $\vec{Px}_{n x 1}$ & ? \\
 \hline
\end{tabular}
\renewcommand{\arraystretch}{1}}
 \caption{}
 \end{table}
Given to minimize the inital investment.\\
Objective Function,
\begin{align}
Z = \min_{b1,\vec{x}} b_1 + 1^T \vec{x}
\end{align}
Constraints,
\begin{align}
&\vec{b} \succeq 0   \\
&\vec{x} \succeq 0 \\
&\vec{w} + \vec{P x} \succeq \vec{e} \\
&b_{t+1} = (1+ \rho)b_t - w_t .
\end{align}
Solution of the above objective function with the constraints is obtained by CVXPY.\\
CVXPY code, 
\begin{lstlisting}
https://github.com/gadepall/EE5606-optimization/codes/opt_2.py
\end{lstlisting}
\begin{lstlisting}
https://github.com/gadepall/EE5606-optimization/codes/expense_stream _data.py
\end{lstlisting}
The optimal initial investment Z = 177.51
\item We are tasked with designing a box shaped with width $w$, height $h$, and depth $d$. We are given that the total wall area is at most 200 square units, the total floor area is at most 60 square units, and the aspect ratios (h/w ans d/w) are at least 0.8 and at most 1.2. Formulate an optimization program to solve for the dimensions h,w and d that results in a box of the largest possible volume, and implement in CVXPY.\\
\solution\\
Given 
\begin{table}[H]
 \centering
 \resizebox{\columnwidth}{!}{
 \begin{tabular}{ |c|c|c| } 
 \hline
 Description & Parameter & Value\\
 \hline
 Width & w & ? \\
 \hline
Height & h & ? \\
 \hline
 Depth &d & ? \\
 \hline
 Volume & v & ? \\
 \hline
 Wall area  & 2(wh+hd) & $\leq$ 200 \\
 \hline
 Floor area  & dw &$ \leq$ 60 \\
 \hline
 Aspect ratio = h/w and d/w & $a_1$ and $a_2$ & $\leq$ 1.2 and $\geq $0.8 \\
 \hline
\end{tabular}}
 \caption{}
 \end{table}

Let $\vec{w}$ be the optimization variables vector.
\begin{align}
\vec{w} = \myvec{w\\h\\d}
\end{align}
Objective is to maximize the volume of the box. \\
\begin{align}
{v=whd} \nonumber
\end{align}

Objective function is defined as,
\begin{align}
&z= \max_{w,h,d} (whd)
\end{align}
constraints are,
\begin{align}
& {wh\leq 200}\\
&{ dw \leq 60}\\
&{0.8\leq a_1\leq 1.2}\\
&{0.8\leq a_2 \leq 1.2}
\end{align}
Writing constraints in the matrix form, 
\begin{align}
&\vec{w}^\text{T}\vec{P_1}\vec{w} \leq q_1\\
& \myvec{w&h&d}\myvec{0&1&0\\0&0&0\\0&1&0}\myvec{w\\h\\d} \leq 100\\
&\vec{w}^\text{T}\vec{P_2}\vec{w} \leq q_2\\
& \myvec{w&h&d}\myvec{0&0&1\\0&0&0\\0&0&0}\myvec{w\\h\\d} \leq 60\\
&\vec{P_3}\vec{w}\leq \vec{q_3}\\
&\myvec{-0.8&-1&0\\1&-1.2&0\\-0.8&0&-1\\1&0&-1.2} \myvec{w\\h\\d} \leq \myvec{0\\0\\0}
 \end{align}
Solving the objective function and constrains by cvxpy.\\
cvxpy code,
\begin{lstlisting}
https://github.com/gadepall/EE5606-optimization/codes/opt_3.py
\end{lstlisting}
The maximum optimum value is 387.29 cubic units.\\
Optimum w = 7.74, h = 6.45, d = 7.74 units. 
\item A fictional company AccessApple  \&  Co. produces three types of covers for Apple products:
one for iPod, one for iPad, and another for iPhone. The company's production facilities are
such that if we devote the entire production to iPod covers, we can produce 6000 of them
in a day. If we devote the entire production to iPhone covers or iPad covers, we can
produce 5000 or 3000 of them, respectively, in one day.
The production schedule is one work week of 5 working days, and the week's production
must be stored before distribution. Storing 1000 iPod covers (packaging included) takes up
40 cubic feet of space. Storing 1000 iPhone covers (packaging included) takes up 45 cubic
feet of space, and storing 1000 iPad covers (packaging included) takes up 210 cubic feet of
space. The total storage space available is 6000 cubic feet.
Due to a commercial agreement, AccessApple \& Co. has to deliver at least 5000 iPod
covers, and 4000 iPad covers per week in order to strengthen the products’ diffusion.
The marketing department estimates that the weekly demand for iPod covers, iPhone, and
iPad covers does not exceed 10000 and 15000 , and 8000 units, respectively. Therefore the
company does not want to produce more than these numbers.
Finally, the net profits in USD for each iPod cover, iPhone cover, and iPad cover are USD 4,
USD 6 and USD 10 respectively.
The aim is to determine a weekly production schedule that maximizes the total net
profit. \\

\solution \\
Consider,\\
x= proportion of time devoted each day to iPod cover production,\\
y= proportion of time devoted each day to iPhone cover production,\\
z= proportion of time devoted each day to iPad cover production.\\
\begin{table}[H]
 \centering
 \resizebox{\columnwidth}{!}{
 \begin{tabular}{ |c|c|c| } 
 \hline
 Description & Parameter & Value  \\
 \hline
 Proportion of time devoted & $\vec{x}$ & $\myvec{x\\y\\z}$  \\ 
 \hline
Production per day & $\vec{p}_p $ & $\myvec{6,000\\ 5,000\\ 3,000} $ \\
 \hline
 Storage per 1000 units & $\vec{s}$ & $\myvec{40\\ 45\\ 210} $ \\ 
 \hline
 minium requirement & $\vec{m}_r$ & $\myvec{5,000\\0\\4,000} $  \\
 \hline
 Maximum Production & $\vec{m}_p$ & $\myvec{10,000\\15,000\\8,000}$ \\
 \hline
 Profit & $\vec{p}$ & $\myvec{4\\6\\10} $ \\
 \hline
 profit for week  & $\vec{c}$ & $\myvec{120000\\15000\\15000} $ \\
 \hline
\end{tabular}}
 \caption{}
 \end{table}
 The elements of $ \vec{c}$ are the product of production per day, profit and no.of workings day in a week for ipod, iphone and ipad respectively.\\
Objective function,
\begin{align}
z = \max_\vec{x} \vec{c}^T\vec{x}\\
z = \min_\vec{x} (- \vec{c}^T\vec{x})
\end{align}
constrains,
\begin{align}
x+y+z \leq 1\\
1200x+1125y+3150 \leq 6000\\
30,0000x \geq 5000\\
15,000z \geq 4000\\
30,000x \leq 10,000\\
25,000y \leq 15,000\\
15,000z \leq 8,000\\
\vec{x}\succeq 0
\end{align}
Putting all the constrains in the form$\vec{P}\vec{x}=\vec{q}$,
\begin{align}
\myvec{1&1&1\\1200&1125&3150\\-30,000&0&0\\0&0&-15,000\\30,000&0&0\\0&25,000&0\\0&0&15,000\\-1&0&0\\0&-1&0\\0&0&-1}\vec{x}=\myvec{1\\6,000\\-5,000\\-4,000\\10,000\\15,000\\8,000\\0\\0\\0}
\end{align}
Solving the objective function and constrains by cvxpy.\\
cvxpy code,
\begin{lstlisting}
https://github.com/gadepall/EE5606-optimization/codes/opt_4.py
\end{lstlisting}
The maximum weekly profit= 1,45,000 USD.\\
Optimum time devoted, $\vec{x} = \myvec{0.1667\\ 0.4129\\ 0.4203}$.
\end{enumerate}
\end{document}


