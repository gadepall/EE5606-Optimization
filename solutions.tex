\documentclass[journal,12pt,twocolumn]{IEEEtran}
%
\usepackage{setspace}
\usepackage{gensymb}
%\doublespacing
\singlespacing
\usepackage{float}
%\usepackage{graphicx}
%\usepackage{amssymb}
%\usepackage{relsize}
\usepackage[cmex10]{amsmath}
%\usepackage{amsthm}
%\interdisplaylinepenalty=2500
%\savesymbol{iint}
%\usepackage{txfonts}
%\restoresymbol{TXF}{iint}
%\usepackage{wasysym}
\usepackage{amsthm}
%\usepackage{iithtlc}
\usepackage{mathrsfs}
\usepackage{txfonts}
\usepackage{stfloats}
\usepackage{bm}
\usepackage{cite}
\usepackage{cases}
\usepackage{subfig}
%\usepackage{xtab}
\usepackage{longtable}
\usepackage{multirow}
%\usepackage{algorithm}
%\usepackage{algpseudocode}
\usepackage{enumitem}
\usepackage{mathtools}
\usepackage{tikz}
\usepackage{circuitikz}
\usepackage{verbatim}
%\usepackage{tfrupee}
\usepackage[breaklinks=true]{hyperref}
%\usepackage{stmaryrd}
\usepackage{tkz-euclide} % loads  TikZ and tkz-base
%\usetkzobj{all}
\usepackage{listings}
    \usepackage{color}                                            %%
    \usepackage{array}                                            %%
    \usepackage{longtable}                                        %%
    \usepackage{calc}                                             %%
    \usepackage{multirow}                                         %%
    \usepackage{hhline}                                           %%
    \usepackage{ifthen}                                           %%
  %optionally (for landscape tables embedded in another document): %%
    \usepackage{lscape}     
\usepackage{multicol}
\usepackage{chngcntr}
%\usepackage{enumerate}

%\usepackage{wasysym}
%\newcounter{MYtempeqncnt}
\DeclareMathOperator*{\Res}{Res}
%\renewcommand{\baselinestretch}{2}
\renewcommand\thesection{\arabic{section}}
\renewcommand\thesubsection{\thesection.\arabic{subsection}}
\renewcommand\thesubsubsection{\thesubsection.\arabic{subsubsection}}

\renewcommand\thesectiondis{\arabic{section}}
\renewcommand\thesubsectiondis{\thesectiondis.\arabic{subsection}}
\renewcommand\thesubsubsectiondis{\thesubsectiondis.\arabic{subsubsection}}

% correct bad hyphenation here
\hyphenation{op-tical net-works semi-conduc-tor}
\def\inputGnumericTable{}                                 %%

\lstset{
%language=C,
frame=single, 
breaklines=true,
columns=fullflexible
}
%\lstset{
%language=tex,
%frame=single, 
%breaklines=true
%}

\begin{document}
%


\newtheorem{theorem}{Theorem}[section]
\newtheorem{problem}{Problem}
\newtheorem{proposition}{Proposition}[section]
\newtheorem{lemma}{Lemma}[section]
\newtheorem{corollary}[theorem]{Corollary}
\newtheorem{example}{Example}[section]
\newtheorem{definition}[problem]{Definition}
%\newtheorem{thm}{Theorem}[section] 
%\newtheorem{defn}[thm]{Definition}
%\newtheorem{algorithm}{Algorithm}[section]
%\newtheorem{cor}{Corollary}
\newcommand{\BEQA}{\begin{eqnarray}}
\newcommand{\EEQA}{\end{eqnarray}}
\newcommand{\define}{\stackrel{\triangle}{=}}

\bibliographystyle{IEEEtran}
%\bibliographystyle{ieeetr}


\providecommand{\mbf}{\mathbf}
\providecommand{\pr}[1]{\ensuremath{\Pr\left(#1\right)}}
\providecommand{\qfunc}[1]{\ensuremath{Q\left(#1\right)}}
\providecommand{\sbrak}[1]{\ensuremath{{}\left[#1\right]}}
\providecommand{\lsbrak}[1]{\ensuremath{{}\left[#1\right.}}
\providecommand{\rsbrak}[1]{\ensuremath{{}\left.#1\right]}}
\providecommand{\brak}[1]{\ensuremath{\left(#1\right)}}
\providecommand{\lbrak}[1]{\ensuremath{\left(#1\right.}}
\providecommand{\rbrak}[1]{\ensuremath{\left.#1\right)}}
\providecommand{\cbrak}[1]{\ensuremath{\left\{#1\right\}}}
\providecommand{\lcbrak}[1]{\ensuremath{\left\{#1\right.}}
\providecommand{\rcbrak}[1]{\ensuremath{\left.#1\right\}}}
\theoremstyle{remark}
\newtheorem{rem}{Remark}
\newcommand{\sgn}{\mathop{\mathrm{sgn}}}
\providecommand{\abs}[1]{\left\vert#1\right\vert}
\providecommand{\res}[1]{\Res\displaylimits_{#1}} 
\providecommand{\norm}[1]{\left\lVert#1\right\rVert}
%\providecommand{\norm}[1]{\lVert#1\rVert}
\providecommand{\mtx}[1]{\mathbf{#1}}
\providecommand{\mean}[1]{E\left[ #1 \right]}
\providecommand{\fourier}{\overset{\mathcal{F}}{ \rightleftharpoons}}
%\providecommand{\hilbert}{\overset{\mathcal{H}}{ \rightleftharpoons}}
\providecommand{\system}{\overset{\mathcal{H}}{ \longleftrightarrow}}
	%\newcommand{\solution}[2]{\textbf{Solution:}{#1}}
\newcommand{\solution}{\noindent \textbf{Solution: }}
\newcommand{\cosec}{\,\text{cosec}\,}
\providecommand{\dec}[2]{\ensuremath{\overset{#1}{\underset{#2}{\gtrless}}}}
\newcommand{\myvec}[1]{\ensuremath{\begin{pmatrix}#1\end{pmatrix}}}
\newcommand{\mydet}[1]{\ensuremath{\begin{vmatrix}#1\end{vmatrix}}}
%\numberwithin{equation}{section}
%\numberwithin{equation}{subsection}
%\numberwithin{problem}{section}
%\numberwithin{definition}{section}
\makeatletter
\@addtoreset{figure}{problem}
\makeatother

\let\StandardTheFigure\thefigure
\let\vec\mathbf
%\renewcommand{\thefigure}{\theproblem.\arabic{figure}}
\renewcommand{\thefigure}{\theproblem}
%\setlist[enumerate,1]{before=\renewcommand\theequation{\theenumi.\arabic{equation}}
%\counterwithin{equation}{enumi}


%\renewcommand{\theequation}{\arabic{subsection}.\arabic{equation}}

\def\putbox#1#2#3{\makebox[0in][l]{\makebox[#1][l]{}\raisebox{\baselineskip}[0in][0in]{\raisebox{#2}[0in][0in]{#3}}}}
     \def\rightbox#1{\makebox[0in][r]{#1}}
     \def\centbox#1{\makebox[0in]{#1}}
     \def\topbox#1{\raisebox{-\baselineskip}[0in][0in]{#1}}
     \def\midbox#1{\raisebox{-0.5\baselineskip}[0in][0in]{#1}}

\title{Convex-Optimization}
\maketitle
Get cvxpy codes from
\begin{lstlisting}
https://github.com/gadepall/EE5606-Optimization/codes
\end{lstlisting}
Get latex-tikz codes from 
\begin{lstlisting}
https://github.com/gadepall/EE5606-Optimization
\end{lstlisting}
\begin{enumerate}
\item There are three food items available, corn, milk, and bread. The table contains,
the cost per serving, the amount of Vitamin A per serving, and the number of
calories per serving for each food item. Also, there are restrictions on the total
number of calories (between 2000 and 2250) and the total amount of Vitamin A
(between 5000 and 50,000) intake in the diet.The goal of the diet problem is to
select a set of food items that will satisfy a set of daily nutritional requirement
at minimum cost. The maximum number of servings is 10 per food item. Table 1 shows all the content per serving.\\
\begin{table}[H]
 \centering
 \resizebox{\columnwidth}{!}{
 \begin{tabular}{ |c|c|c|c| } 
 \hline
 Food & Cost ($\vec{c}$) & Vitamin A ($\vec{v}$) & calories ($\vec{u}$) \\
 \hline
Corn & 0.18 USD & 107 & 72\\
 \hline
 Milk & 0.23 USD & 500 & 121\\
 \hline
 Wheat Bread & 0.05 USD &0 &65\\
 \hline
\end{tabular}}
 \caption{}
 \end{table}
\solution 
Consider the vector $\vec{c}$ as cost vector, vector $\vec{v}$ representing the vitamin A and vector $\vec{u}$ representing calories.\\
Let $\vec{x}$ be the variable vector.\\
\begin{align}
&\vec{c} = \myvec{0.18\\0.23\\0.05}\\
& \vec{v} = \myvec{107\\500\\0}\\
& \vec{u} = \myvec{72\\121\\65}\\
&\vec{x} = \myvec{x\\y\\z}
\end{align}
Objective function 
\begin{align}
z = \min_\vec{x} \vec{c}^T\vec{x}
\end{align}
Constraints
\begin{align}
&5000 \leq 107x+500y+0z \leq 50000\\
&2000 \leq 72x+121y+65z \leq 2250\\
&0 \leq x \leq 10\\
&0 \leq y \leq 10\\
&0 \leq z \leq 10
\end{align}
Writing all the constraints in the matrix form
\begin{align}
&\vec{p}\vec{x} = \vec{q}\\
&\myvec{107&500&0\\-107&-500&0\\72&121&65\\-72&-121&-65\\1&0&0\\0&1&0\\0&0&1\\-1&0&0\\0&-1&0\\0&0&-1}\vec{x}=\myvec{50000\\ -5000\\ 2250\\ -2000\\ 10\\ 10\\ 10\\ 0\\ 0\\ 0}
\end{align}
By providing the objective function and constraints to cvxpy, we get the optimal cost (z) and optimal no.of servings $(\vec{x})$.\\
cvxpy code,
\begin{lstlisting}
https://github.com/gadepall/EE5606-optimization/codes/opt_1.py
\end{lstlisting}
From cvxpy, we get
\begin{align}
& z= 3.16  \text{USD}\\
& \vec{x} = \myvec{2\\10\\10}
\end{align}
\end{enumerate}

\end{document}


